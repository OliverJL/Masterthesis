\chapter{The First Chapter}
\label{cha:1}
%The following section will deliver the relevant information regarding memory management an ASLR.
%
%% EVP (Enhanced Virus Protection) – available on
%all 64-bit AMD CPUs (starting in 2003)
%\cite{guide2017amd64}
%
%% XD Bit (Execute Disable Bit) – available with
%Pentium 4 "Prescott" core (starting in 2004)1
%\cite{guide2017intel}
%If the execute-disable bit of a memory page is set, that page can be used only as data. An attempt to execute code
%from a memory page with the execute-disable bit set causes a page-fault exception.
%
%\section{Memory Management}
%Modern Operation System implement an abstraction layer.
%
%Load elf: sections -> segments.
%-------------------------------------------
%Virtual Address space (linearer Adressraum)
%--------------------------------------------
%Linear Address Space
%From a user perspective, the address space is a at linear address space but pre-
%dictably, the kernel's perspective is very different. The address space is split into
%two parts, the userspace part which potentially changes with each full context switch
%and the kernel address space which remains constant. The location of the split is
%determined by the value of PAGE\_OFFSET which is at 0xC0000000 on the x86. This
%means that 3GiB is available for the process to use while the remaining 1GiB is
%always mapped by the kernel. The linear virtual address space as the kernel sees it
%is illustrated in Figure 4.1.
%\cite{gorman2004linuxvmmgr}
%
%
%Up Down Kernel/User space
%--------------------------












\cite{cowan2000buffer}
\cite{marco2014effectiveness}
\cite{kerrigan2012study}
\cite{guide2017intel}
\cite{wojtczuk2011following}
\cite{alt2015entropy}
\cite{thompsonrandomness}
\cite{celesti2010improving}
\cite{mueller@bsi1}
\cite{mueller@bsi2}
\cite{marco2013preventing}


%\section{The First Topic of the Chapter}
%First comes the introduction to this topic.
%
%\lipsum[55]
%
%\subsection{An item}
%Please don't abuse enumerations: short enumerations shouldn't use
%``\verb|itemize|'' or ``\texttt{enumerate}'' environments.
%So \emph{never write}: 
%\begin{quote}
%  The Eiffel tower has three floors:
%  \begin{itemize}
%  \item the first one;
%  \item the second one;
%  \item the third one.
%  \end{itemize}
%\end{quote}
%But write:
%\begin{quote}
%  The Eiffel tower has three floors: the first one, the second one, and the
%  third one.
%\end{quote}
%
%\section{A Second Topic}
%\lipsum[64]
%
%\subsection{Another item}
%\lipsum[56-57]
%
%\section{Conclusion}
%The final section of the chapter gives an overview of the important results
%of this chapter. This implies that the introductory chapter and the
%concluding chapter don't need a conclusion.
%
%\lipsum[66]

%%% Local Variables: 
%%% mode: latex
%%% TeX-master: "thesis"
%%% End: 
