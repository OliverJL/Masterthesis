\chapter{The Next Chapter}
\label{cha:2}

\section{Operating System Virtualization with the Xen-Hypervisor}

Virtualization of operating systems has increased continuously over the recent years and layed the foundations for new resource management concepts and business models like f.e. cloud hosting. Regarding server virtualization, many organizations exceed rates of 75\% \cite{gartnervmmarket}. By executing an operating system as a virtual machine (VM), improvements in several areas can be achieved. As instances of even different operating systems can be run parallel on the same host machine, system capacity in terms of processing power, volatile and persistent memory can be utilized more efficient. A VM can even be assigned resources dynamically at runtime, without the need of a restart. Further, backup \& recovery strategies can be simplified, as a VM's physical presentation at any certain state can be preserved in files. This even enables to move a running VM from one to another server in the same resource pool with virtually no service interruption \cite{migratevms}. Also the process of providing a new VM instance can be slimmed down drastically. Instead of being required to execute an operating system setup for a particular hardware composition, a once created VM master image can be cloned and configured. Depending on the applied technology, VMs can be setup up on basis of operating system setup images which are targeted for installation directly on hardware, without any further modification required. In other words, the virtual machine monitor or hypervisor enables the execution of regular proprietary or open source operating systems in a virtual environment. \\
According to a study of Gartner from 2016, the market for x86 server virtualization infrastructure software is partitioned among few competitors, led by VMWare, Microsoft, Oracle and Red Hat. While VMWare and Microsoft offer proprietary solutions, Oracle and Red Hat adopted the open-source Xen-hypervisor technology. For cloud infrastructures, the Xen-Hypervisor remains the most widely used architecture for public infrastructure as a service (IaaS) cloud provider. This fact is predominantly attributable to Amazon's utilization of the Xen-Hypervisor for its cloud solutions "Amazon Web Services" \cite{bittman2016magic}.
\\~\\
Hier muss noch ein Uebergang hin
\\


- Open Source
- Amazon
- Verbreitung 
- HVM / PVM
- Architektur

------------------------------------
\cite{everspaugh2014not}
Second,aVM  can be repeatedly executed froma ?xed image which is the default for amazon
Dort auch: infos zu regualr boot/snapshot/masterimage
------------------------------------



However, while there are many reasons to utilize virtual machines, most of the standard security practices of operating systems are based on assumptions that hold true for physical machines, but don't translate immediately into the domain of virtualized machines \cite{kerrigan2012study}. A virtual hosted Linux server f.e. will never receive a user triggered event during the boot process, which is a critical period within the initialization of the internal Linux PRNG. Further, new virtual machine instances are usually provided by cloning a master image, instead of running an installation setup, leading to a higher degree of homogeneity among virtual machine instances. While provisioning time and maintenance benefit from this practice, it may be regarded critical. If this approach f.e. is applied as the standard process of a cloud hosting provider, it`s customers may be delivered a vulnerable system from the outset, since one customer might gain insights regarding another customers system, just by analyzing his own. These insights can be used to exploit VM reset vulnerabilities which take advantage of the reuse of  operating system snapshots, so called snapshot replay. Thomas Ristenpart et. al. describe this type of attack and apply it on TSL implementations with disastrous results. Relevant reasons they identified are the exposure of randomness, as well as the inability to find sufficient entropy in virtual systems environments \cite{ristenpart2010good, ristenpart2009hey}. \\~\\
According to a study of Gartner from 2016, the market for x86 server virtualization infrastructure software is partitioned among few competitors, led by VMWare, Microsoft, Oracle and Red Hat. While VMWare and Microsoft offer proprietary solutions, Oracle and Red Hat adopted the open-source Xen-hypervisor technology. For cloud infrastructures, the Xen-hypervisor remains the most widely used architecture for public infrastructure as a service (IaaS) cloud provider. This fact is predominantly attributable to Amazon's utilization of the Xen-hypervisor for its cloud solutions "Amazon Web Services" \cite{bittman2016magic}. 

Similarly, the market of common server operating systems can be divided into proprietary and open-source solutions. While Microsoft dominates the fraction of proprietary systems with its Windows Server series, Linux based systems have a total share of \~37\%. Within this share, \~39\% of servers dedicated to host websites etc. are running an Ubuntu Distribution \cite{statsharelinux} . \\~\\


\section{Random Number Generation by Linux x64 Kernel 4.4}


Intro:
valid kernel for x64  4.4. (16.04 LTS)
MISP 

 

arch:
Intel IvyBridge x86-Prozessorgeneration
TRNG
 \cite guide2017intel
RDRAND returns random numbers that are supplied by a cryptographically secure, deterministic random bit generator DRBG. The DRBG is designed to meet the NIST SP 800-90A standard.
RDSEED
Non-deterministic random bit generator
NIST SP 800-90B \& C
\cite{guide2017intel}
Jegliche Nutzung der RDRAND-Instruktion kann verhindert werden, wenn der Kern mit der
Kommandozeilenoption ?nordrand? gestartet wird.
Es ist zu beachten, dass RDRAND von einem Hypervisor im Sinne eines VM-Exits abgefangen
und ver�ndert beziehungsweise nicht an die CPU weitergegeben werden kann.
\cite{mueller@bsi2}

Diff: kernel/Userspace request

\begin{comment}
Aktuelle Arbeiten zeigen, dass gerade im Bereich von Headless-Systemen, die beim
ersten Systemstart Zufallszahlen f�r die Erzeugung von Schl�sseln ben�tigen, zu
wenig Entropie vorhanden ist. Auch wenn von den Interrupts nicht viel Entropie zu
erwarten ist, sollte die Verwendung der Interrupt-Ereignisse dieses Problem etwas
abmildern.
\cite{mueller@bsi2}
\end{comment}


\begin{figure}[H]
%\centering


%\makeatother


	
	\begin{tikzpicture}[scale=0.8, every node/.style={scale=0.8},font=\sffamily,>=triangle 45]
	\tikzstyle{sum} = [draw, shape=circle, node distance=1.5cm, line width=1pt, minimum width=1.25em]	
	%\Huge
	\def\N{7}  % Number of Flip-Flops minus one
	\def\BW{1.8} % Byte Width

	% colors
	\definecolor{cjh}{HTML}{CFD2D4}
	\definecolor{cjl}{HTML}{DD6262}
	\definecolor{cch}{HTML}{F7A100}
	\definecolor{ccl}{HTML}{0288CF}
	\definecolor{cep32}{HTML}{64FE2E}
	\definecolor{cep1}{HTML}{1893D4}
	\definecolor{cep0}{HTML}{A9FD2A}
	\definecolor{cirq}{HTML}{DB15E5}
	\definecolor{cip}{HTML}{FFF200}

	% jiffies
	\node [shape=dff,fill=cjh] (jiff7) at ($ 1.0*(0,0) $) {03};
	\node [shape=dff,fill=cjh] (jiff6) at ($ 1.0*(1,0) $) {F7};
	\node [shape=dff,fill=cjh] (jiff5) at ($ 1.0*(2,0) $) {E4};
	\node [shape=dff,fill=cjh] (jiff4) at ($ 1.0*(3,0) $) {CC};			
	\node [shape=dff,fill=cjl] (jiff3) at ($ 1.0*(4,0) $) {05};
	\node [shape=dff,fill=cjl] (jiff2) at ($ 1.0*(5,0) $) {A4};
	\node [shape=dff,fill=cjl] (jiff1) at ($ 1.0*(6,0) $) {97};
	\node [shape=dff,fill=cjl] (jiff0) at ($ 1.0*(7,0) $) {80};	
	\node[above=1mm of jiff3] (ljiffies) {\large $in:$jiffies (8 byte)};		

	% jiffies low XOR c_high - XOR nodes
	\node [sum, below=4.0cm of jiff0, draw] (xjlch0) {};
	\node [sum, below=4.5cm of jiff1, draw] (xjlch1) {};	
	\node [sum, below=5.0cm of jiff2, draw] (xjlch2) {};
	\node [sum, below=5.5cm of jiff3, draw] (xjlch3) {};	
	\node [rotate=45] at (xjlch0) (plus) {{\footnotesize$+$}};
	\node [rotate=45] at (xjlch1) (plus) {{\footnotesize$+$}};	
	\node [rotate=45] at (xjlch2) (plus) {{\footnotesize$+$}};
	\node [rotate=45] at (xjlch3) (plus) {{\footnotesize$+$}};
		
	% cycles
	\node [shape=dff,fill=cch] (cycl7) at ($ 1.0*(10,0) $) {FF};
	\node [shape=dff,fill=cch] (cycl6) at ($ 1.0*(11,0) $) {EE};
	\node [shape=dff,fill=cch] (cycl5) at ($ 1.0*(12,0) $) {DD};
	\node [shape=dff,fill=cch] (cycl4) at ($ 1.0*(13,0) $) {CC};			
	\node [shape=dff,fill=ccl] (cycl3) at ($ 1.0*(14,0) $) {BB};
	\node [shape=dff,fill=ccl] (cycl2) at ($ 1.0*(15,0) $) {AA};
	\node [shape=dff,fill=ccl] (cycl1) at ($ 1.0*(16,0) $) {99};
	\node [shape=dff,fill=ccl] (cycl0) at ($ 1.0*(17,0) $) {88};
	\node[above=1mm of cycl3] (lcycles) {\large $in:$cycles counter (8 byte)};	

	%c_high
	\node [shape=dff,fill=cch, below=2cm of jiff7, draw] (chigh3) {FF};
	\node [shape=dff,fill=cch, below=2cm of jiff6, draw] (chigh2) {EE};
	\node [shape=dff,fill=cch, below=2cm of jiff5, draw] (chigh1) {DD};
	\node [shape=dff,fill=cch, below=2cm of jiff4, draw] (chigh0) {CC};
	\node[above=3mm of chigh2] (lchigh) {\large c\_high (4 byte)};			

	%j_high
	\node [shape=dff,fill=cjh, below=2cm of cycl7, draw] (jiffh3) {FF};
	\node [shape=dff,fill=cjh, below=2cm of cycl6, draw] (jiffh2) {EE};
	\node [shape=dff,fill=cjh, below=2cm of cycl5, draw] (jiffh1) {DD};
	\node [shape=dff,fill=cjh, below=2cm of cycl4, draw] (jiffh0) {CC};
	\node[above=3mm of jiffh1] (ljhigh) {\large j\_high (4 byte)};			

	% jiffies high XOR irq - XOR nodes
	\node [sum, below=1.0cm of jiffh3, draw] (xjhi3) {};	
	\node [rotate=45] at (xjhi3) (plus) {{\footnotesize$+$}};
	\node [sum, below=1.0cm of jiffh2, draw] (xjhi2) {};	
	\node [rotate=45] at (xjhi2) (plus) {{\footnotesize$+$}};
	\node [sum, below=1.0cm of jiffh1, draw] (xjhi1) {};	
	\node [rotate=45] at (xjhi1) (plus) {{\footnotesize$+$}};
	\node [sum, below=1.0cm of jiffh0, draw] (xjhi0) {};	
	\node [rotate=45] at (xjhi0) (plus) {{\footnotesize$+$}};
			
	%irq
	\node [shape=dff,fill=cirq, below=1cm of xjhi3, draw] (irq3) {FF};
	\node [shape=dff,fill=cirq, below=1cm of xjhi2, draw] (irq2) {EE};
	\node [shape=dff,fill=cirq, below=1cm of xjhi1, draw] (irq1) {DD};
	\node [shape=dff,fill=cirq, below=1cm of xjhi0, draw] (irq0) {CC};
	\node[left=0mm of irq3] (lirq) {\shortstack{\large $in:$irq\\(4 byte)}};	
	
	% cycles low XOR irq - XOR nodes
	\node [sum, below=8.5cm of cycl3, draw] (xcli3) {};
	\node [sum, below=8.0cm of cycl2, draw] (xcli2) {};	
	\node [sum, below=7.5cm of cycl1, draw] (xcli1) {};		
	\node [sum, below=7.0cm of cycl0, draw] (xcli0) {};			
	\node [rotate=45] at (xcli3) (plus) {{\footnotesize$+$}};
	\node [rotate=45] at (xcli2) (plus) {{\footnotesize$+$}};	
	\node [rotate=45] at (xcli1) (plus) {{\footnotesize$+$}};
	\node [rotate=45] at (xcli0) (plus) {{\footnotesize$+$}};	
	
	% entropy pool	
	\node [shape=dff,fill=cep32, below=11cm of jiff5, draw] (entp15) {CC};		
	\node [shape=dff,fill=cep32, below=11cm of jiff4, draw] (entp14) {CC};
	\node [shape=dff,fill=cep32, below=11cm of jiff3, draw] (entp13) {CC};
	\node [shape=dff,fill=cep32, below=11cm of jiff2, draw] (entp12) {CC};
	\node [shape=dff,fill=cep32, below=11cm of jiff1, draw] (entp11) {CC};
	\node [shape=dff,fill=cep32, below=11cm of jiff0, draw] (entp10) {CC};
	\node [shape=dff,fill=cep32, right=0cm of entp10, draw] (entp9) {CC};	
	\node [shape=dff,fill=cep32, right=0cm of entp9, draw] (entp8) {CC};
	\node [shape=dff,fill=cep1, right=0cm of entp8, draw] (entp7) {CC};
	\node [shape=dff,fill=cep1, below=11cm of cycl6, draw] (entp6) {XX};	
	\node [shape=dff,fill=cep1, below=11cm of cycl5, draw] (entp5) {XX};	
	\node [shape=dff,fill=cep1, below=11cm of cycl4, draw] (entp4) {XX};	
	\node [shape=dff,fill=cep0, below=11cm of cycl3, draw] (entp3) {XX};	
	\node [shape=dff,fill=cep0, below=11cm of cycl2, draw] (entp2) {XX};	
	\node [shape=dff,fill=cep0, below=11cm of cycl1, draw] (entp1) {XX};	
	\node [shape=dff,fill=cep0, below=11cm of cycl0, draw] (entp0) {XX};	
	\node[above=1mm of entp12] (lentp) {\large $in/out:$fast pool (16 byte)};	

	%% XOR entp
	% ch / jl 
	\node [sum, below=2.5cm of xjlch3, draw] (xentp7) {};	
	\node [rotate=45] at (xentp7) (plus) {{\footnotesize$+$}};
	\node [sum, below=2.5cm of xjlch2, draw] (xentp6) {};	
	\node [rotate=45] at (xentp6) (plus) {{\footnotesize$+$}};
	\node [sum, below=2.5cm of xjlch1, draw] (xentp5) {};	
	\node [rotate=45] at (xentp5) (plus) {{\footnotesize$+$}};
	\node [sum, below=2.5cm of xjlch0, draw] (xentp4) {};	
	\node [rotate=45] at (xentp4) (plus) {{\footnotesize$+$}};									
	% irq/cycl 
	\node [sum, above=1.0cm of entp3, draw] (xentp3) {};	
	\node [rotate=45] at (xentp3) (plus) {{\footnotesize$+$}};
	\node [sum, above=1.0cm of entp2, draw] (xentp2) {};	
	\node [rotate=45] at (xentp2) (plus) {{\footnotesize$+$}};
	\node [sum, above=1.0cm of entp1, draw] (xentp1) {};	
	\node [rotate=45] at (xentp1) (plus) {{\footnotesize$+$}};
	\node [sum, above=1.0cm of entp0, draw] (xentp0) {};	
	\node [rotate=45] at (xentp0) (plus) {{\footnotesize$+$}};		
	% ip
	\node [sum, below=1.0cm of entp15, draw] (xentp15) {};	
	\node [rotate=45] at (xentp15) (plus) {{\footnotesize$+$}};
	\node [sum, below=1.0cm of entp14, draw] (xentp14) {};	
	\node [rotate=45] at (xentp14) (plus) {{\footnotesize$+$}};
	\node [sum, below=1.0cm of entp13, draw] (xentp13) {};	
	\node [rotate=45] at (xentp13) (plus) {{\footnotesize$+$}};
	\node [sum, below=1.0cm of entp12, draw] (xentp12) {};	
	\node [rotate=45] at (xentp12) (plus) {{\footnotesize$+$}};		
	\node [sum, below=1.0cm of entp11, draw] (xentp11) {};	
	\node [rotate=45] at (xentp11) (plus) {{\footnotesize$+$}};
	\node [sum, below=1.0cm of entp10, draw] (xentp10) {};	
	\node [rotate=45] at (xentp10) (plus) {{\footnotesize$+$}};
	\node [sum, below=1.0cm of entp9, draw] (xentp9) {};	
	\node [rotate=45] at (xentp9) (plus) {{\footnotesize$+$}};
	\node [sum, below=1.0cm of entp8, draw] (xentp8) {};	
	\node [rotate=45] at (xentp8) (plus) {{\footnotesize$+$}};	

	\node [shape=dff,fill=cip, below=1cm of xentp15, draw] (ip7) {XX};	
	\node [shape=dff,fill=cip, below=1cm of xentp14, draw] (ip6) {XX};
	\node [shape=dff,fill=cip, below=1cm of xentp13, draw] (ip5) {XX};
	\node [shape=dff,fill=cip, below=1cm of xentp12, draw] (ip4) {XX};
	\node [shape=dff,fill=cip, below=1cm of xentp11, draw] (ip3) {XX};
	\node [shape=dff,fill=cip, below=1cm of xentp10, draw] (ip2) {XX};
	\node [shape=dff,fill=cip, below=1cm of xentp9, draw] (ip1) {XX};
	\node [shape=dff,fill=cip, below=1cm of xentp8, draw] (ip0) {XX};
	\node[below=1mm of ip4] (lip) {\large $in:$ip (8 byte)};	

	%%%%%% LINES >>
	
	% jiffies  -> c_low - lines
	\draw [->] (jiff3.south) -- (xjlch3.north);
	\draw [->] (jiff2.south) -- (xjlch2.north);	
	\draw [->] (jiff1.south) -- (xjlch1.north);		
	\draw [->] (jiff0.south) -- (xjlch0.north);		

	% c_high -> xjlch
	\draw [->] (chigh3.south)|- (xjlch3.west);
	\draw [->] (chigh2.south)|- (xjlch2.west);
	\draw [->] (chigh1.south)|- (xjlch1.west);
	\draw [->] (chigh0.south)|- (xjlch0.west);			

	% jiffies high -> j_high
	\draw [->] (jiff7.south) -- (jiffh3.north);
	\draw [->] (jiff6.south) -- (jiffh2.north);	
	\draw [->] (jiff5.south) -- (jiffh1.north);		
	\draw [->] (jiff4.south) -- (jiffh0.north);		

	% cycles_high -> c_high
	\draw [->] (cycl7.south) -- (chigh3.north);
	\draw [->] (cycl6.south) -- (chigh2.north);	
	\draw [->] (cycl5.south) -- (chigh1.north);		
	\draw [->] (cycl4.south) -- (chigh0.north);		
	
	% cycles_low -> c_high
	\draw [->] (cycl3.south) -- (xcli3.north);
	\draw [->] (cycl2.south) -- (xcli2.north);	
	\draw [->] (cycl1.south) -- (xcli1.north);		
	\draw [->] (cycl0.south) -- (xcli0.north);	

	% jiffies high -> j_high
	\draw [->] (jiffh3.south) -- (xjhi3.north);
	\draw [->] (jiffh2.south) -- (xjhi2.north);	
	\draw [->] (jiffh1.south) -- (xjhi1.north);		
	\draw [->] (jiffh0.south) -- (xjhi0.north);
		
	% xjhi -> irq
	\draw [<->] (xjhi3.south) -- (irq3.north);
	\draw [<->] (xjhi2.south) -- (irq2.north);	
	\draw [<->] (xjhi1.south) -- (irq1.north);		
	\draw [<->] (xjhi0.south) -- (irq0.north);
	
	% xjlch -> xentp
	\draw [->] (xjlch3.south) -- (xentp7.north);
	\draw [->] (xjlch2.south) -- (xentp6.north);	
	\draw [->] (xjlch1.south) -- (xentp5.north);		
	\draw [->] (xjlch0.south) -- (xentp4.north);

	% irq -> xcli
	\draw [->] (irq3.south) |- (xcli3.west);
	\draw [->] (irq2.south) |- (xcli2.west);	
	\draw [->] (irq1.south) |- (xcli1.west);		
	\draw [->] (irq0.south) |- (xcli0.west);

	% xentp -> entp
	\draw [<->] (xentp7.south) |- ($(entp7.north)!1/2!(entp7.north |- xentp7.south)$) coordinate (C1) -| (entp7.north);			
	\draw [<->] (xentp6.south) |- ($(entp6.north)!1/2!(entp6.north |- xentp6.south)$) coordinate (C2) -| (entp6.north);		
	\draw [<->] (xentp5.south) |- ($(entp5.north)!1/2!(entp5.north |- xentp5.south)$) coordinate (C3) -| (entp5.north);				
	\draw [<->] (xentp4.south) |- ($(entp4.north)!1/2!(entp4.north |- xentp4.south)$) coordinate (C4) -| (entp4.north);	
	
	% xcli -> xentp
	\draw [->] (xcli3.south) -- (xentp3.north);
	\draw [->] (xcli2.south) -- (xentp2.north);	
	\draw [->] (xcli1.south) -- (xentp1.north);		
	\draw [->] (xcli0.south) -- (xentp0.north);	

	% xcli -> xentp
	\draw [<->] (xentp3.south) -- (entp3.north);
	\draw [<->] (xentp2.south) -- (entp2.north);	
	\draw [<->] (xentp1.south) -- (entp1.north);		
	\draw [<->] (xentp0.south) -- (entp0.north);
	
	% xentp -> entp
	\draw [<->] (xentp15.north) -- (entp15.south);
	\draw [<->] (xentp14.north) -- (entp14.south);
	\draw [<->] (xentp13.north) -- (entp13.south);
	\draw [<->] (xentp12.north) -- (entp12.south);
	\draw [<->] (xentp11.north) -- (entp11.south);
	\draw [<->] (xentp10.north) -- (entp10.south);
	\draw [<->] (xentp9.north) -- (entp9.south);
	\draw [<->] (xentp8.north) -- (entp8.south);
	
	% ip -> xentp
	\draw [->] (ip7.north) -- (xentp15.south);
	\draw [->] (ip6.north) -- (xentp14.south);
	\draw [->] (ip5.north) -- (xentp13.south);
	\draw [->] (ip4.north) -- (xentp12.south);
	\draw [->] (ip3.north) -- (xentp11.south);
	\draw [->] (ip2.north) -- (xentp10.south);
	\draw [->] (ip1.north) -- (xentp9.south);
	\draw [->] (ip0.north) -- (xentp8.south);
	
	%%%%%% Legende 
\begin{scope}
	\node [shape=dff,fill=cip, below=10mm of entp6, draw] (f8) {};
	\node [sum, right=5mm of f8, draw] (xf8) {};	
	\node [rotate=45] at (xf8) (plus) {{\footnotesize$+$}};	
	\node [shape=dff,fill=cirq, right=7mm of xf8, draw] (c3) {};
	\draw [->] (f8.east) -- (xf8.west);
	\draw [<->] (xf8.east) -- (c3.west);
	\node[black, below=3mm of xf8, align=left] (lf81) {XOR-Operation applied on};	
	\node [shape=dff,fill=cip, right=0mm of lf81, draw] (rf8) {};
	\node[black, right=0mm of rf8, align=left] (lf82) {and};	
	
	\node [shape=dff,fill=cirq, below=22mm of entp6, draw] (yf8) {};
	\node[black, right=0mm of yf8, align=left] (lf83) {, result stored in};
	\node [shape=dff,fill=cirq, right=0mm of lf83, draw] (zf8) {};	
	\draw[black] ([xshift=-5mm, yshift=3mm ]f8.north west) rectangle ([xshift=23mm, yshift=-3mm]zf8.south east);	
\end{scope}	

\begin{scope}
\large
%\node [draw, align=center] {Text\\und Text};
\node[black, below=10mm of ip7, align=left] (lghdlbl) {\textbf{Label}};	
\node[black, right=28mm of lghdlbl.west, anchor=west, align=left] (lghddirr) {\textbf{Direction}};
\node[black, right=18mm of  lghddirr.west, anchor=west, align=left] (lghddesc) {\textbf{Description}};

%\node[black, below=10mm of ip4, align=left] (lgjif) {\textbf{cycles counter}};	
\node[black, below=4mm of lghdlbl.west, anchor=west, align=left] (lgcc) {cycles counter};	
\node[black, below=4mm of lgcc.west, anchor=west] (lgjif) {jiffies};	
\node[black, below=4mm of lgjif.west, anchor=west] (lgirq) {irq};	
\node[black, below=4mm of lgirq.west, anchor=west] (lgip) {ip};
\node[black, below=4mm of lgip.west, anchor=west] (lgentp) {fast pool};

\node[black, right=33mm of lgcc.west, align=left] (lgdircc) {in};	
\node[black, right=33mm of lgjif.west, align=left] (lgdirjif) {in};	
\node[black, right=33mm of lgirq.west, align=left] (lgdirirq) {in};
\node[black, right=33mm of lgip.west, align=left] (lgdirip) {in};
\node[black, right=30mm of lgentp.west, align=center] (lgdientp) {in/out};


\node[black, right=24mm of lgcc, align=left] (lgcctxt) {Number of CPU-cycles since system startup.};	
\node[black, below=4mm of lgcctxt.west, anchor=west] (lgjiftxt) {Number of ticks occured since system startup.};	
\node[black, below=4mm of lgjiftxt.west, anchor=west] (lgirqtxt) {Interrupt Request Code};
\node[black, below=4mm of lgirqtxt.west, anchor=west] (lgiptxt) {Instruction Pointer};
\node[black, below=4mm of lgiptxt.west, anchor=west] (lgentptxt) {Pseudo entropy pool, initializing input- \& nonblocking Pool};

\draw[black] ([xshift=-5mm, yshift=3mm ]lghdlbl.north west) rectangle ([xshift=5mm, yshift=-3mm]lgentptxt.south east);
	
\end{scope}
\end{tikzpicture}
\caption{Processing of input parameters by func. 'add\_interrupt\_randomness' before applying fast\_mix / mix\_pool\_bytes operations (valid for 64-bit Kernel only)} \label{fig:add-int-rnd-chart}
\end{figure}


%\begin{mdframed}
%\begin{tabularx}{\columnwidth}{XXl}
%\begin{tabularx}{\textwidth}{ll}
%%	\caption{Description of input parameters proccessed by func. 'add\_interrupt\_randomness'}
%%	\label{tab:add-int-rnd-desc}\\
%	\textbf{jiffies}&Igel\\
%	\textbf{cycles counter}&Dienstag\\
%	\textbf{irq}&\\
%	\textbf{ip}&\textit{Instruction Pointer}
%	\caption{Description of input parameters proccessed by func. 'add\_interrupt\_randomness'}
%\end{tabularx}
%\centering
%\begin{table}[H]
%	\begin{tabular}{ll}
%	\textbf{Jiffies}&Nr. of ticks occured since system startup.\\
%	\textbf{Cycles Counter}&Nr. of CPU-cycles since system startup.\\
%	\textbf{irq}&Interrupt Request \\
%	\textbf{ip}&\textit{Instruction Pointer}
%	\end{tabular}
%	\caption{Description of input parameters proccessed by func. 'add\_interrupt\_randomness'}	
%\end{table}

% jiffies . Incremented for each timer interrupt.
% . also known as \textit{Time Stamp Counter}.

%	\cite{kernlrandmc}
		
%\end{mdframed}


%\begin{tabularx}{\columnwidth}{XXl}
%	jiffies&Schnecke&Igel\\
%	cycles counter&Hier ist ein langes Wort Hier ist ein langes Wort Hier ist ein langes Wort Hier ist ein langes Wort &Dienstag\\
%	irq&&\\
%	ip&&
%\end{tabularx}
%
%
%
%\begin{mdframed}
%\begin{description}
%	\item [jiffies] sadfasdfsadf
%	\item [cycles counter] asdfasdfasdf
%	\item [irq] asdfasdfasdf	
%	\item [ip] 
%\end{description}
%\cite{kernlrandmc}	
%\end{mdframed}




\section{Conclusion}
The final section of the chapter gives an overview of the important results
of this chapter. This implies that the introductory chapter and the
concluding chapter don't need a conclusion.

\lipsum[66]

%%% Local Variables: 
%%% mode: latex
%%% TeX-master: "thesis"
%%% End: 
