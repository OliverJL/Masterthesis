\chapter{The Next Chapter}
\label{cha:2}

\section{Operating System Visualization with the Xen Hypervisor}

Virtualization of operating systems has increased continuously over the recent years and layed the foundations for new resource management concepts and business models like f.e. cloud hosting. Regarding server virtualization, many organizations exceed rates of 75\% \cite{gartnervmmarket}. By executing an operating system as a virtual machine, improvements in several areas can be achieved.
As several instances of even different operating systems can be run parallel on the same host machine,
system resources in terms of processing power, volatile and persistent memory can be utilized more efficient. A 

 the scalability of system resources in terms of processing power, volatile and persistent memory. 

 like scalability and more efficient utilization of hardware resources, backup \& disaster recovery strategy, accelerated provision of new system instances etc. can be achieved. Depending on the applied technology, such virtual machines can be setup up on basis of operating system setup images which are targeted for installation directly on hardware, without any further modification required. In other words, the virtual machine monitor or hypervisor enables the execution of regular commercial or open source operating systems in a virtual environment. 



However, while there are many reasons to utilize virtual machines, most of the standard security practices of operating systems are based on assumptions that hold true for physical machines, but don't translate immediately into the domain of virtualized machines \cite{kerrigan2012study}. A virtual hosted Linux server f.e. will never receive a user triggered event during the boot process, which is a critical period within the initialization of the internal Linux PRNG. Further, new virtual machine instances are usually provided by cloning a master image, instead of running an installation setup, leading to a higher degree of homogeneity among virtual machine instances. While provisioning time and maintenance benefit from this practice, it may be regarded critical. If this approach f.e. is applied as the standard process of a cloud hosting provider, it`s customers may be delivered a vulnerable system from the outset, since one customer might gain insights regarding another customers system, just by analyzing his own. These insights can be used to exploit VM reset vulnerabilities which take advantage of the reuse of  operating system snapshots, so called snapshot replay. Thomas Ristenpart et. al. describe this type of attack and apply it on TSL implementations with disastrous results. Relevant reasons they identified are the exposure of randomness, as well as the inability to find sufficient entropy in virtual systems environments \cite{ristenpart2010good, ristenpart2009hey}. \\~\\
According to a study of Gartner from 2016, the market for x86 server virtualization infrastructure software is partitioned among few competitors, led by VMWare, Microsoft, Oracle and Red Hat. While VMWare and Microsoft offer proprietary solutions, Oracle and Red Hat adopted the open-source Xen-hypervisor technology. For cloud infrastructures, the Xen-hypervisor remains the most widely used architecture for public infrastructure as a service (IaaS) cloud provider. This fact is predominantly attributable to Amazon's utilization of the Xen-hypervisor for its cloud solutions "Amazon Web Services" \cite{bittman2016magic}. Similarly, the market of common server operating systems can be divided into proprietary and open-source solutions. While Microsoft dominates the fraction of proprietary systems with its Windows Server series, Linux based systems have a total share of \~37\%. Within this share, \~39\% of servers dedicated to host websites etc. are running an Ubuntu Distribution \cite{statsharelinux} . \\~\\



- Open Source
- Amazon
- Verbreitung 
- HVM / PVM
- Architektur


\section{Conclusion}
The final section of the chapter gives an overview of the important results
of this chapter. This implies that the introductory chapter and the
concluding chapter don't need a conclusion.

\lipsum[66]

%%% Local Variables: 
%%% mode: latex
%%% TeX-master: "thesis"
%%% End: 
