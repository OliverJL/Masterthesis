\chapter{Random Number Generation in Linux x64 Kernel 4.4}
\label{cha:3}

\section{General}\label{sec:lrng-general}

Random numbers are required by a variety of security related applications. An obvious purpose is encryption or signing of data, like f.e. disk encryption, Transport Socket Layer (TLS) or digital signatures of emails. In this case random numbers are used to generate cryptographically secure keys. Concepts like Stack Canaries (aka. Stack Guards) or Address Space Layout Randomization (ASLR) rely purely on the nondeterministic character of random numbers, applying them to make certain types of attacks on vulnerable applications much more difficult. Overall, the effectiveness of such applications depends significantly on the quality i.e. unpredictably of random numbers, provided by an operating system. \\
In the following, a comprehensive introduction to the Linux Random Number Generator (LRNG) with focus on analyzed concepts and components of this thesis will be given. It refers to Linux kernel v.4.4, which is currently used in several major Long Term Support distributions like Ubuntu Server 16.04 LTS. If facts are valid for Intel x86/x64 platforms only, those will be noted as such.
As the analysis of random numbers i.e. their quality in terms of entropy will be analyzed compliant to an approach (\cite{turan2018nist}) recommended by the National Institute of Standards and Technology the corresponding terminology will be used.\\
In general, many modern CPU-architectures implement an on-chip True Random Number Generators. Starting with 'IvyBridge' architecture, Intel provides two instructions accessing directly those entropy sources. The \textsf{RDRAND} instruction returns random numbers that are supplied by a cryptographically secure, deterministic random bit generator which is designed to meet the NIST SP 800-90A standard. If an application or operating system design insists on applying in it's own Pseudo Random Number Generator (PRNG) it may apply the \textsf{RDSEED} instruction which is compliant to NIST SP 800-90B \& C standard \cite{guide2017intel} for initializing i.e. seeding its state.
Availability of these TRNGs does not automatically imply their application. If a kernel is started 
with paramter \textsf{nordrand} (on x86/x64 architectures), the \textsf{RDRAND} instruction will have no effect. Further, if an operating system is run as a virtual machine, those instructions can by trapped by the hypervisor instead of being passed to the CPU \cite{mueller@bsi2}. Thus, the analysis in this thesis does not take on-chip entropy pools into account, but refers to the pure Linux Pseudo Random Number Generator.

\section{Linux Pseudo Random Number Generator Architecture Overview}\label{sec:lrng-arch}
The Linux Pseudo Random Number Generator's (PRNG) responsibility is for providing random numbers to requesting consumers. To ensure a certain quality of randomness the implementation's activity is  divided into four major tasks:   

%\begin{figure}
\begin{itemize}
	\item Capturing and processing of input data which is finally stored in entropy pools or buffers
	\item Assessing and health checking the level of entropy in each pool in unit 'bits of entropy'
	\item Diverting incoming entropy to specific pools, depending on a pool's state
	\item Releasing random numbers to a consumer 
\end{itemize}
%	\caption{Major tasks of the Linux PRNG}
%\end{figure}
Depending on the type of consumer, there is a fundamental difference in the way random numbers can be requested. Applications or services running in user mode access those via different interfaces than logic running with kernel privileges does. Also, the source of entropy varies vastly, depending on the consumer type. The security components analyzed in this thesis are involved in the generation of new processes. This task is accomplished by code running on kernel level, hence the initialization and management of entropy sources for this consumer type are in focus of this introduction, but partly overlap with those dedicated to user mode consumers.
\\~\\
To generate random numbers, the Linux PRNG needs to fall back on events delivered by hardware. 
Capturing and processing of several types of hardware events starts immediately after the kernel has been loaded into memory and is given execution by the boot loader. With the occurrence of each type of hardware event, pre-processing routines (more or less sophisticated transformations) are applied on the delivered input parameters. Those may be as simple as a single XOR- or ADD-operation to more complex but lean transformations, partly reinvolving previously calculated results. The outcome of these operations is finally assigned to entropy pools. An entropy pool is a data structure located in kernel memory. Beside storing the generated random numbers, it is also aware of its entropy state. Depending on its purpose, an entropy pool delivers data to a consumer interface, or is meant to enrich other pools. In total, in the Linux kernel exist three full-featured entropy pools which will be explained in the following. Some hardware events are processed by a so called fast\_pool, which must be seen as a pre-processing stage instead of a full-featured entropy pool \cite{mueller@bsi2}. Additionally, part of the initialization process is the set up of a 16 byte vector, which is assigned once during system startup. This vector is involved as a seed for conditioning i.e. more sophisticated transformations like MD5 or SHA2, applied on kernel level consumers' requests of random numbers. Hence the development of this vector is of particular interest for this thesis. Figure \ref{fig:linux-prng-arch} illustrates all major components involved in the generation and management of random numbers via the Linux PRNG. The following sections will explain their dedication and interaction more briefly.

\begin{figure}[H]
	\centering
	\begin{tikzpicture}[scale=0.8, every node/.style={scale=0.8},font=\sffamily,>=triangle 45]
\tikzstyle{sum} = [draw, shape=circle, node distance=1.5cm, line width=1pt, minimum width=1.25em]	
%\Huge
\def\N{7}  % Number of Flip-Flops minus one
\def\BW{1.8} % Byte Width

% colors
\definecolor{rc-input}{HTML}{E8B50C}
\definecolor{gr-input}{HTML}{E8B50C}
\definecolor{rc-input-pool}{HTML}{0DFF11}
\definecolor{rc-user-pool}{HTML}{52ABE8}
\definecolor{rc-kernel-pool}{HTML}{0D38FF}
\definecolor{rc-fast-pool-col}{HTML}{CFD2D4}
\definecolor{rc-rnd-int-sec-vect-col}{HTML}{4B0CE8}
%4B0CE8

%4B0CE8
%B8FFF7

\definecolor{cjh}{HTML}{CFD2D4}
\definecolor{cjl}{HTML}{DD6262}
\definecolor{cch}{HTML}{F7A100}
\definecolor{ccl}{HTML}{0288CF}
\definecolor{cep32}{HTML}{64FE2E}
\definecolor{cep1}{HTML}{1893D4}
\definecolor{cep0}{HTML}{A9FD2A}
\definecolor{cirq}{HTML}{DB15E5}
\definecolor{cip}{HTML}{FFF200}

% blocing pool
%\node [shape=dff,fill=cjh, width=30mm] (jiff7) at ($ 1.0*(0,0) $) {blocking pool};
% rect rc-blocking-pool
\node (rc-blocking-pool) [rectangle, rounded corners, fill=rc-user-pool!60, draw, minimum width=35mm, minimum height=10mm, anchor= south west] at (0,0) {\shortstack{blocking pool\\/dev/random}};
% rect get_random_int/long
\node (rc-getrandomintlong) [rectangle, rounded corners, right=10mm of rc-blocking-pool, fill=rc-kernel-pool!50, draw, minimum width=35mm, minimum height=10mm, anchor= west] {\shortstack{get\_random\_int/long\\kernel mode only}};		
% rect rc-nonblocking-pool
\node (rc-nonblocking-pool) [rectangle, rounded corners, right=50mm of rc-blocking-pool, fill=rc-user-pool!60, draw, minimum width=35mm, minimum height=10mm, anchor= west] {\shortstack{non-blocking pool\\/dev/urandom}};	
% rect rc-input-pool	
\node (rc-input-pool) [rectangle, rounded corners, below right=23mm and 10mm of rc-blocking-pool, fill=rc-input-pool!50, draw, minimum width=35mm, minimum height=10mm, anchor= west] {input pool};
% rc-input-pool -> rect rc-blocking-pool
\draw [->] (rc-input-pool.west) -| (rc-blocking-pool);	
% rc-input-pool -> rect rc-nonblocking-pool
\draw [->] (rc-input-pool.east) -| (rc-nonblocking-pool);	
% rect rc-rnd-int-sec	
\node (rc-rnd-int-sec) [rectangle, rounded corners, above=4mm of rc-input-pool.north, fill=rc-rnd-int-sec-vect-col!30, draw, minimum width=45mm, minimum height=3mm] {random int secret vector};
% rc-nonblocking-pool -> rc-rnd-int-sec
\draw [<-] (rc-rnd-int-sec.east) -| ($(rc-nonblocking-pool.west)!1/2!(rc-nonblocking-pool.west -| rc-rnd-int-sec.east)$) coordinate (C4) -| (rc-nonblocking-pool.west);
% rc-rnd-int-sec -> rect get_random_int/long
\draw [->] (rc-rnd-int-sec.north) -- (rc-getrandomintlong.south);
\node[above right=5mm and -8mm of rc-getrandomintlong.west] (lentp) {\large interfaces / conditioning pools };
\begin{pgfonlayer}{background}
% background - pre-processing
\path (rc-blocking-pool.west |- rc-blocking-pool.north)+(-0.8,0.9) node (a) {};
\path (rc-nonblocking-pool.east |- rc-nonblocking-pool.south)+(0.4,-0.4) node (b) {};
\path [fill=rc-user-pool!40,rounded corners, draw=black!50, dashed] (a) rectangle (b); 
\end{pgfonlayer}
		

%	\draw [->] (rc-nonblocking-pool.west) -| (rc-rnd-int-sec.north);

% rect rc-interrupt-inp
\node (rc-interrupt-inp) [rectangle, rounded corners, below=70mm of rc-blocking-pool.west, fill=rc-input, draw, minimum 
width=30mm, minimum height=10mm, anchor= west] {\shortstack{interrupt input}};
% rect rc-device-inp	
\node (rc-device-inp) [rectangle, rounded corners, right=3mm of rc-interrupt-inp.east, fill=rc-input, draw, minimum width=30mm, minimum height=10mm, anchor= west] {\shortstack{device input}};
% rect rc-key-inp	
\node (rc-key-inp) [rectangle, rounded corners, right=3mm of rc-device-inp.east, fill=rc-input, draw, minimum width=30mm, minimum height=10mm, anchor= west] {\shortstack{keyborad input}};
% rect rc-disk-inp	
\node (rc-disk-inp) [rectangle, rounded corners, right=3mm of rc-key-inp.east, fill=rc-input, draw, minimum width=20mm, minimum height=10mm, anchor= west] {\shortstack{disk input}};

% rect rc-fast-pool	
\node (rc-fast-pool) [rectangle, rounded corners, above right=22mm and 10mm of rc-interrupt-inp.west, fill=rc-fast-pool-col, draw, minimum width=30mm, minimum height=10mm, anchor= west] {\shortstack{fast pool}};
% rc-interrupt-inp -> rc-fast-pool
\draw [->] (rc-interrupt-inp.north) -- (rc-fast-pool.south);
% rc-fast-pool -> rc-input-pool
\draw [->] (rc-fast-pool) -- (rc-input-pool);
% rc-device-inp -> rc-input-pool
\draw [->] (rc-device-inp) -- (rc-input-pool);
\node[above right=4mm and -6mm of rc-fast-pool.west] (lentp) {\large post-processing};
\begin{pgfonlayer}{background}
% background - pre-processing
\path (rc-fast-pool.west |- rc-fast-pool.north)+(-0.8,0.7) node (a) {};
\path (rc-fast-pool.east |- rc-fast-pool.south)+(0.4,-0.4) node (b) {};
\path [fill=rc-fast-pool-col!40,rounded corners, draw=black!50, dashed] (a) rectangle (b); 
\end{pgfonlayer}


rc-fast-pool-col

% rect rc-timer-inp	
\node (rc-timer-inp) [rectangle, rounded corners, above right=22mm and 8mm of rc-key-inp.west, fill=rc-input, draw, minimum width=35mm, minimum height=10mm, anchor= west] {timer input};	
% rc-key-inp -> rc-timer-inp
\draw [->] (rc-key-inp) -- (rc-timer-inp);
% rc-disk-inp -> rc-timer-inp
\draw [->] (rc-disk-inp) -- (rc-timer-inp);
% rc-timer-inp -> rc-input-pool
\draw [->] (rc-timer-inp) -- (rc-input-pool);

\node[below right=5mm and 8mm of rc-device-inp.west] (lentp) {\large entropy input sources};
\begin{pgfonlayer}{background}
% background - entropy input sources
\path (rc-interrupt-inp.west |- rc-interrupt-inp.north)+(-0.4,0.4) node (a) {};
\path (rc-disk-inp.east |- rc-disk-inp.south)+(0.4,-0.8) node (b) {};
\path [fill=gr-input!40,rounded corners, draw=black!50, dashed] (a) rectangle (b); 
\end{pgfonlayer}


\end{tikzpicture}		

	\caption{Input sources and entropy pools, -vectors, interfaces of the Linux PRNG} 
	\label{fig:linux-prng-arch}
\end{figure}

\section{Input sources of the Linux PRNG}\label{sec:inp-src-lprng}
In best case, the generation of random numbers is conducted by dedicated devices in terms of True Random Number Generators. The Linux RNG implementation allows to involve TRNGs in two ways: 
\begin{itemize}
	\item \textit{Instruction Delegation:} Requests from consumers are directly translated into instructions using architecture dependent on-chip features as described in \ref{sec:lrng-general}.
	This approach bypasses the entire logic of the Linux PRNG and relies completely on the TRNG.
	\item \textit{TRNG/PRNG Hybrid Mode:} Random numbers are managed via the regular PRNG logic. Instead of transformed hardware events, the entropy pools are filled with data, provided by a TRNG device via a special driver interface.
\end{itemize}
For this thesis, both of these options are out of scope. Instead, the process of generating random numbers based on common hardware events is analyzed. As an outcome, it turned out that just two types of hardware events were able to provide input data, while the reaming did not contribute anything at all, since the operating system was run as a virtual machine. Despite this, also those should be mentioned for the sake of completeness.

\subsection{Timing Variance}
For each hardware event type a data structure \textsf{timer\_rand\_state} is maintained tracking the occurrence of events. Depending on the event type, this information will used to enrich the 
input pool via passing it's value to \textsf{add\_timer\_randomness} and/or to discard the event parameter if a minimum interval has not pass yet. 

\subsection{Input Randomness}\label{sec:inp-rnd}
Input events comprise any kind of parameters delivered by user input devices like a keyboard or a mouse. A keyboard the usually delivers the keycode, a mouse the coordinate of the cursor position. However the devices need to be connected locally \cite{mueller@bsi2}. Since this does not apply a an operating system run as a virtual machine, \textsf{add\_input\_randomness} is never called and further also no timer input is delivered to \textsf{add\_timer\_randomness} by this type of event.

\subsection{Disk Randomness}\label{sec:disk-rnd}
Per disk i.e. block device the amount of seek time of a block layer request  \textsf{add\_disk\_randomness} will contribute to the input pool via a call of \textsf{add\_timer\_randomness}. However, if a virtual machine is hosted on a solid state drives timing variance of events is to low and constant. Hence also this event type does not deliver any noise for this setup.

\subsection{Device Randomness}\label{sec:dev-rnd}
Function \textsf{add\_device\_randomness} is called just one, in a very early phase, when the kernel starts to initialize device drivers. In the applied virtual environment this event type's parameters where limited on BIOS/UEFI related information like manufacture name, vendor-id and the serial number. This information commonly is static. While at least some parameters may be individual on a hardware installation, the Xen-Hypervisor allows to use an identical BIOS/UEFI image for arbitrary virtual machines.   

\subsection{Interrupt Randomness}\label{sec:int-rnd}
Via the concept of interrupts, system components are enabled to request an interrupt of the system's current activity, to deliver or demand information. When an interrupt request (IRQ) is signaled, the system will halt at defined state, exchange data with requesting unit and finally or immediately proceed execution. The intention of a request is communicated via an IRQ-code. These codes are well known to the operating system, as predefined by convention and set by the BIOS. IRQ 0 f.e., signals a periodic occurring timer event. As illustrated in \ref{fig:linux-prng-arch}, detouring the fast pool, interrupt input initializes the input pool. As mentioned in \ref{sec:inp-rnd} and \ref{sec:disk-rnd}, disk- and input events don't contribute any noise at all, while entropy provided by \textsf{add\_device\_randomness} is assumed to be poor. In other words, the development of the input entropy pools depends significantly on the parameters provided by \textsf{add\_interrupt\_randomness} and is hence of particular interest, since this pool is finally responsible to initialize the \textsf{random\_int\_secret} vector, which is applied as seed for each request triggered by a kernel process. TODOC
%if ((fast_pool->count < 64) && !time_after(now, fast_pool->last + HZ))
% erwaehnen das add_interrupt_randomness abbricht wenn fast_pool->last + HZ

\section{Entropy Pools, Pre-Processing and Vectors}
An entropy pool is a complex data structure dedicated to store generated random values as well as its own state regarding the currently achieved entropy level. In total, in Linux Kernel 4.4 are existing three pools based on data structure \textsf{entropy\_store}, named blocking-pool, non-blocking-pool and input-pool (see \ref{fig:linux-prng-arch}) \cite{kernlrandmc}. The blocking- and non-blocking-pool are so called output pools, directly accessibly via applications or services running in \gls{usrmd}. By contrast, code running in \gls{knlmd} uses the kernel internal functions \textsf{get\_random\_int} i.e. \textsf{get\_random\_long} to retrieve random values. Those consumer interfaces have a conditioning behavior in common, hence also called conditioning pool. When random numbers are requested, a cryptographic conditioning procedure is applied on the values to be delivered. This practice has advantages regarding the pools own security, as it avoids any exposure of the pools internal state \cite{kernlrandmc}. Further, the conditioning procedure's transformation (MD5/SHA) achieves an \gls{aveff}. Algorithms providing this property are characterized by switching the state of each single bit of output having length \textit{[n]} at a probability of close to 0.5 for two input bit words of length \textit{[n]}, even if the deviation of the inputs is just one bit \cite{damasevicius2012energy}. Hence those conditioning procedures have also an increasing impact 
on the entropy of delivered numbers. In the following, these pools, as well as involved components like pre-proccessing stages will be described more briefly, with a focus on kernel internal interfaces.


\subsection{Input Pool}\label{sub:input-pool}
The input pool is a central entropy pool dedicated to accumulate noise from the input sources described in \ref{sec:inp-src-lprng} and distribute it to the output pools. If noise is delivered by interrupt events, it will have been pre-processed by the fast pool. Incoming entropy is integrated into the pool (of length 512 byte) via an operation which mixes obtained values to existing. This procedure is not cryptographic safe, but a compromise between performance and the requirement for a good non-cryptographic hash \cite{kernlrandmc}. With each addition, an entropy counter which tracks the pools internal state is incremented. If a sufficient level is reached, incoming values are passed directly to the output pools, to avoid a waste of entropy. In general the  input pool is the one and only source of entropy for the output pool accessible for user mode consumers. When an output pool 
submits a request, the input pool extracts data from its internal vector. Before  delivery, those values will transformed by a SHA1 conditioning procedure \cite{kernlrandmc}. Indirectly the input pool is responsible for the initialization of the \textsf{random\_int\_secret} vector which is highly involved in the generation of random numbers for kernel consumers (see \ref{sub:rnd-int-sec}).
%  It is not accessible by a consumer directly, but dedicated to develop the output pools during initialization phase and further to feed those if entropy is requested. The input pool is a conditioning pool and hence applies a SHA1 transformation on the extracted data before completing delivery to the requesting output pool . It tracks its internal state by an entropy counter which is incremented when entropy is added i.e. decreased when is extracted. 

\subsection{Blocking- \& Unblocking Pool}\label{sub:blocking-pool}
The blocking- \& unblocking pool are two entropy pools which are accessible to user mode components via the device interfaces \textsf{/dev/random} i.e. \textsf{/dev/urandom} having an internal length of 128 byte. Both pools are exclusively feed by the input pool (see \ref{sub:input-pool}). The fundamental difference is their behaviour in case of a poor internal state i.e. a low degree of entropy in the internal vector. Request made to \textsf{/dev/urandom} are responded immediately. Hence also a large amount of random numbers can be obtained by a user mode consumer \cite{lacharme2012linux}. Contrary, for this scenario, \textsf{/dev/random} will deny delivery until 
the internal state achieved a healthy level of entropy. Thus this interface should not be used for 
applications demanding a large amount of entropy or having time critical behaviour.


\subsection{Fast Pools}
According to \ref{sec:lrng-arch}, fast pools are no full-featured entropy pools, but have the purpose of a pre-processing stage, dedicated to initialize the input pool. As illustrated in \ref{fig:linux-prng-arch}, fast pools are feed by noise delivered by interrupt event input. They are declared by a simple data structure, which is able to store 16 byte of gained random values beside few state information, as shown in \ref{fig:fast-pool}. This data structure is instantiated and maintained per CPU, available to the virtualized operating system.
%TODO: Hier noch ein Verweis rein, das diese relativ fix sind (gelesen in mark..)??
 
 \begin{lstlisting}[caption=Declaration of the fast\_pool structure in Linux kernel 4.4 \cite{kernlrandmc}, captionpos=b,
 xrightmargin=.3\textwidth,
 xleftmargin=.3\textwidth,
 label=lst:fastspool]
 struct fast_pool {
    __u32 pool[4];
    unsigned long last;
    unsigned short reg_idx;
    unsigned char count;
 };
 \end{lstlisting}
 
 For this thesis, the number of available CPUs per virtual machine has been configured to one, hence the terminology will refer to a single instance in the following. By concept, in Linux Kernel 4.4, interrupt randomness is the sole source of random noise available to rise up the fast pool, which is a contributor to the input pool. Observations of the experimental setup for this thesis concluded that beside interrupt events any contribution has occurred by input- or disk events. In other words, the development of the fast pool depends completely on the input and pre-processing conducted by \textsf{add\_interrupt\_randomness} and hence will be described more briefly.\\
 For each call of function \textsf{add\_interrupt\_randomness}, several parameters are processed to transform the 16 byte vector of the fast pool. Thereby just two parameters are directly passed by the caller. Surprisingly, parameter \textsf{irq\_code} (explained in \ref{sec:int-rnd}), is processed, while the second, \textsf{irq\_flags} is not used at all, for unknown reasons (see \ref{lst:add-interrupt-randomnes-c}). Beside this input, three more runtime values are applied on pre-proccessing procedures within the function. Some of these are also involved in the conditioning procedure of entropy interfaces for kernel consumers (see \ref{sub:get-rnd-int-long}). In the following these three noise sources are explained to give a basic understanding regarding the entropy they are able to provide.
 
%\ref{sec:int-rnd}
 
 \begin{itemize}
 	\item \textbf{Cycles Counter [8 byte]} The cycles counter (see 'cycles' in  \ref{lst:add-interrupt-randomnes-c}) is a high-resolution counter of 8 byte length (on x64 plattforms), used to receive CPU timing information. It is also referred as Time Stamp Counter and is invokable via the \textsf{RDTSC} instruction \cite{guide2017intel}. Cycles Counter is managed by the CPU which monotonically increments the time-stamp counter model-specific register (MSR) every clock cycle and resets it to 0 whenever	the processor is reseted. As mentioned in TODO the results retrieved by the calling \textsf{RDTSC} instruction may not be directly passed to the CPU but deliver emulation based results, depending on the Xen configuration setting \textsf{tsc\_mode} \cite{xentscmode} for a particular virtual machine. For the given Virtual Environment Setup of this thesis, per default, the cycles counter value has been provided by the underlying CPU to both virtual machines.
 	
	\item \textbf{Instruction Pointer [8 byte]} On x64 platforms, the instruction pointer (see 'ip' in \ref{lst:add-interrupt-randomnes-c}) holds the 64-bit offset of the next
	instruction to be executed, which is kept in CPU-register \textsf{RIP}. This architecture also support a technique called \textsf{RIP}-relative addressing. Using this technique, the effective address is determined by adding a displacement to the \textsf{RIP} of the next instruction \cite{guide2017intel}.
	
	\item \textbf{Jiffies [8 byte]} Jiffies (see 'now' in \ref{lst:add-interrupt-randomnes-c}) represent a kernel-internal value which is incremented at each timer interrupt. This counter may be seen as some kind of software clock maintained by the kernel which measures time in jiffies \cite{timeandtimers}. The accuracy of jiffies is determined by the value of the kernel compile-time constant \textsf{HZ} TODO Ref. On system startup, jiffies is initialized to a default value.
	
 \end{itemize}
 
  Before integrating those noise values into the fast pool, several pre-processing transformations are applied. These simple algorithms combine shift- and xor-operations to a non cryptographic safe procedure able to complicate any conclusion on the original input. The current state of the fast pool is contributed as input itself in each pre-processing activity. A detailed overview of the operations integrating the noise delivered by the cycles counter, instruction pointer, irq code and jiffies is illustrated in \ref{fig:add-int-rnd}.
 
 \subsection{Random Int Secret Vector}\label{sub:rnd-int-sec}
 At an early stage of the boot process, a global, kernel internal vector of 64 byte length is initialized once by 
 
\cite{yoo2017analysis}
 
 \ref{sub:input-pool}
 
 \textsf{random\_int\_secret}
 
 
 \subsection{get\_random\_int/\_long}\label{sub:get-rnd-int-long}
 
 
 
 
 
 %MD5-transform
 
 %TODO (nachhaken-soll das rein?)
 
%  in 
% \ref{sub:input-pool}
 
 %enrichment
 
% Pre-processing
%fast_pool
% \textsf{add\_interrupt\_randomness}
%struct fast_pool {
%	__u32		pool[4];
%	unsigned long	last;
%	unsigned short	reg_idx;
%	unsigned char	count;
%};

%
%{sec:int-rnd}
%one per CPU
%XOR 
%solely


%\begin{lstlisting}[float, caption={declares of the fast_pool structure},captionpos=b, xleftmargin=.4\textwidth]
%struct fast_pool {
%__u32		pool[4];
%unsigned long	last;
%unsigned short	reg_idx;
%unsigned char	count;
%};
%\end{lstlisting}




%\textsf{add\_interrupt\_randomness}



% In total there are X types 

% Those will be explained in XXX and YYY.  



%generates random numbers based on occurring hardware events. Those events 


\begin{comment}
Jegliche Nutzung der RDRAND-Instruktion kann verhindert werden, wenn der Kern mit der
Kommandozeilenoption ?nordrand? gestartet wird.
Es ist zu beachten, dass RDRAND von einem Hypervisor im Sinne eines VM-Exits abgefangen
und ver�ndert beziehungsweise nicht an die CPU weitergegeben werden kann.

\end{comment}


%implementation
%
%Intel IvyBridge x86 
%
%
%
%Intro:
%valid kernel for x64  4.4. (16.04 LTS)
%MISP 
%
% 
%
%arch:
%Intel IvyBridge x86-Prozessorgeneration
%TRNG
% \cite guide2017intel
%RDRAND returns random numbers that are supplied by a cryptographically secure, deterministic random bit generator DRBG. The DRBG is designed to meet the NIST SP 800-90A standard.
%RDSEED
%Non-deterministic random bit generator
%NIST SP 800-90B \& C
%\cite{guide2017intel}
%
%
%Diff: kernel/Userspace request

\begin{comment}
Aktuelle Arbeiten zeigen, dass gerade im Bereich von Headless-Systemen, die beim
ersten Systemstart Zufallszahlen f�r die Erzeugung von Schl�sseln ben�tigen, zu
wenig Entropie vorhanden ist. Auch wenn von den Interrupts nicht viel Entropie zu
erwarten ist, sollte die Verwendung der Interrupt-Ereignisse dieses Problem etwas
abmildern.
\cite{mueller@bsi2}
\end{comment}




%\begin{figure}[H]
%\label{fig:fast-pool}
%\centering
%

%\makeatother


	
	\begin{tikzpicture}[scale=0.8, every node/.style={scale=0.8},font=\sffamily,>=triangle 45]
	\tikzstyle{sum} = [draw, shape=circle, node distance=1.5cm, line width=1pt, minimum width=1.25em]	
	%\Huge
	\def\N{7}  % Number of Flip-Flops minus one
	\def\BW{1.8} % Byte Width

	% colors
	\definecolor{cjh}{HTML}{CFD2D4}
	\definecolor{cjl}{HTML}{DD6262}
	\definecolor{cch}{HTML}{F7A100}
	\definecolor{ccl}{HTML}{0288CF}
	\definecolor{cep32}{HTML}{64FE2E}
	\definecolor{cep1}{HTML}{1893D4}
	\definecolor{cep0}{HTML}{A9FD2A}
	\definecolor{cirq}{HTML}{DB15E5}
	\definecolor{cip}{HTML}{FFF200}

	% jiffies
	\node [shape=dff,fill=cjh] (jiff7) at ($ 1.0*(0,0) $) {03};
	\node [shape=dff,fill=cjh] (jiff6) at ($ 1.0*(1,0) $) {F7};
	\node [shape=dff,fill=cjh] (jiff5) at ($ 1.0*(2,0) $) {E4};
	\node [shape=dff,fill=cjh] (jiff4) at ($ 1.0*(3,0) $) {CC};			
	\node [shape=dff,fill=cjl] (jiff3) at ($ 1.0*(4,0) $) {05};
	\node [shape=dff,fill=cjl] (jiff2) at ($ 1.0*(5,0) $) {A4};
	\node [shape=dff,fill=cjl] (jiff1) at ($ 1.0*(6,0) $) {97};
	\node [shape=dff,fill=cjl] (jiff0) at ($ 1.0*(7,0) $) {80};	
	\node[above=1mm of jiff3] (ljiffies) {\large $in:$jiffies (8 byte)};		

	% jiffies low XOR c_high - XOR nodes
	\node [sum, below=4.0cm of jiff0, draw] (xjlch0) {};
	\node [sum, below=4.5cm of jiff1, draw] (xjlch1) {};	
	\node [sum, below=5.0cm of jiff2, draw] (xjlch2) {};
	\node [sum, below=5.5cm of jiff3, draw] (xjlch3) {};	
	\node [rotate=45] at (xjlch0) (plus) {{\footnotesize$+$}};
	\node [rotate=45] at (xjlch1) (plus) {{\footnotesize$+$}};	
	\node [rotate=45] at (xjlch2) (plus) {{\footnotesize$+$}};
	\node [rotate=45] at (xjlch3) (plus) {{\footnotesize$+$}};
		
	% cycles
	\node [shape=dff,fill=cch] (cycl7) at ($ 1.0*(10,0) $) {FF};
	\node [shape=dff,fill=cch] (cycl6) at ($ 1.0*(11,0) $) {EE};
	\node [shape=dff,fill=cch] (cycl5) at ($ 1.0*(12,0) $) {DD};
	\node [shape=dff,fill=cch] (cycl4) at ($ 1.0*(13,0) $) {CC};			
	\node [shape=dff,fill=ccl] (cycl3) at ($ 1.0*(14,0) $) {BB};
	\node [shape=dff,fill=ccl] (cycl2) at ($ 1.0*(15,0) $) {AA};
	\node [shape=dff,fill=ccl] (cycl1) at ($ 1.0*(16,0) $) {99};
	\node [shape=dff,fill=ccl] (cycl0) at ($ 1.0*(17,0) $) {88};
	\node[above=1mm of cycl3] (lcycles) {\large $in:$cycles counter (8 byte)};	

	%c_high
	\node [shape=dff,fill=cch, below=2cm of jiff7, draw] (chigh3) {FF};
	\node [shape=dff,fill=cch, below=2cm of jiff6, draw] (chigh2) {EE};
	\node [shape=dff,fill=cch, below=2cm of jiff5, draw] (chigh1) {DD};
	\node [shape=dff,fill=cch, below=2cm of jiff4, draw] (chigh0) {CC};
	\node[above=3mm of chigh2] (lchigh) {\large c\_high (4 byte)};			

	%j_high
	\node [shape=dff,fill=cjh, below=2cm of cycl7, draw] (jiffh3) {FF};
	\node [shape=dff,fill=cjh, below=2cm of cycl6, draw] (jiffh2) {EE};
	\node [shape=dff,fill=cjh, below=2cm of cycl5, draw] (jiffh1) {DD};
	\node [shape=dff,fill=cjh, below=2cm of cycl4, draw] (jiffh0) {CC};
	\node[above=3mm of jiffh1] (ljhigh) {\large j\_high (4 byte)};			

	% jiffies high XOR irq - XOR nodes
	\node [sum, below=1.0cm of jiffh3, draw] (xjhi3) {};	
	\node [rotate=45] at (xjhi3) (plus) {{\footnotesize$+$}};
	\node [sum, below=1.0cm of jiffh2, draw] (xjhi2) {};	
	\node [rotate=45] at (xjhi2) (plus) {{\footnotesize$+$}};
	\node [sum, below=1.0cm of jiffh1, draw] (xjhi1) {};	
	\node [rotate=45] at (xjhi1) (plus) {{\footnotesize$+$}};
	\node [sum, below=1.0cm of jiffh0, draw] (xjhi0) {};	
	\node [rotate=45] at (xjhi0) (plus) {{\footnotesize$+$}};
			
	%irq
	\node [shape=dff,fill=cirq, below=1cm of xjhi3, draw] (irq3) {FF};
	\node [shape=dff,fill=cirq, below=1cm of xjhi2, draw] (irq2) {EE};
	\node [shape=dff,fill=cirq, below=1cm of xjhi1, draw] (irq1) {DD};
	\node [shape=dff,fill=cirq, below=1cm of xjhi0, draw] (irq0) {CC};
	\node[left=0mm of irq3] (lirq) {\shortstack{\large $in:$irq\\(4 byte)}};	
	
	% cycles low XOR irq - XOR nodes
	\node [sum, below=8.5cm of cycl3, draw] (xcli3) {};
	\node [sum, below=8.0cm of cycl2, draw] (xcli2) {};	
	\node [sum, below=7.5cm of cycl1, draw] (xcli1) {};		
	\node [sum, below=7.0cm of cycl0, draw] (xcli0) {};			
	\node [rotate=45] at (xcli3) (plus) {{\footnotesize$+$}};
	\node [rotate=45] at (xcli2) (plus) {{\footnotesize$+$}};	
	\node [rotate=45] at (xcli1) (plus) {{\footnotesize$+$}};
	\node [rotate=45] at (xcli0) (plus) {{\footnotesize$+$}};	
	
	% entropy pool	
	\node [shape=dff,fill=cep32, below=11cm of jiff5, draw] (entp15) {CC};		
	\node [shape=dff,fill=cep32, below=11cm of jiff4, draw] (entp14) {CC};
	\node [shape=dff,fill=cep32, below=11cm of jiff3, draw] (entp13) {CC};
	\node [shape=dff,fill=cep32, below=11cm of jiff2, draw] (entp12) {CC};
	\node [shape=dff,fill=cep32, below=11cm of jiff1, draw] (entp11) {CC};
	\node [shape=dff,fill=cep32, below=11cm of jiff0, draw] (entp10) {CC};
	\node [shape=dff,fill=cep32, right=0cm of entp10, draw] (entp9) {CC};	
	\node [shape=dff,fill=cep32, right=0cm of entp9, draw] (entp8) {CC};
	\node [shape=dff,fill=cep1, right=0cm of entp8, draw] (entp7) {CC};
	\node [shape=dff,fill=cep1, below=11cm of cycl6, draw] (entp6) {XX};	
	\node [shape=dff,fill=cep1, below=11cm of cycl5, draw] (entp5) {XX};	
	\node [shape=dff,fill=cep1, below=11cm of cycl4, draw] (entp4) {XX};	
	\node [shape=dff,fill=cep0, below=11cm of cycl3, draw] (entp3) {XX};	
	\node [shape=dff,fill=cep0, below=11cm of cycl2, draw] (entp2) {XX};	
	\node [shape=dff,fill=cep0, below=11cm of cycl1, draw] (entp1) {XX};	
	\node [shape=dff,fill=cep0, below=11cm of cycl0, draw] (entp0) {XX};	
	\node[above=1mm of entp12] (lentp) {\large $in/out:$fast pool (16 byte)};	

	%% XOR entp
	% ch / jl 
	\node [sum, below=2.5cm of xjlch3, draw] (xentp7) {};	
	\node [rotate=45] at (xentp7) (plus) {{\footnotesize$+$}};
	\node [sum, below=2.5cm of xjlch2, draw] (xentp6) {};	
	\node [rotate=45] at (xentp6) (plus) {{\footnotesize$+$}};
	\node [sum, below=2.5cm of xjlch1, draw] (xentp5) {};	
	\node [rotate=45] at (xentp5) (plus) {{\footnotesize$+$}};
	\node [sum, below=2.5cm of xjlch0, draw] (xentp4) {};	
	\node [rotate=45] at (xentp4) (plus) {{\footnotesize$+$}};									
	% irq/cycl 
	\node [sum, above=1.0cm of entp3, draw] (xentp3) {};	
	\node [rotate=45] at (xentp3) (plus) {{\footnotesize$+$}};
	\node [sum, above=1.0cm of entp2, draw] (xentp2) {};	
	\node [rotate=45] at (xentp2) (plus) {{\footnotesize$+$}};
	\node [sum, above=1.0cm of entp1, draw] (xentp1) {};	
	\node [rotate=45] at (xentp1) (plus) {{\footnotesize$+$}};
	\node [sum, above=1.0cm of entp0, draw] (xentp0) {};	
	\node [rotate=45] at (xentp0) (plus) {{\footnotesize$+$}};		
	% ip
	\node [sum, below=1.0cm of entp15, draw] (xentp15) {};	
	\node [rotate=45] at (xentp15) (plus) {{\footnotesize$+$}};
	\node [sum, below=1.0cm of entp14, draw] (xentp14) {};	
	\node [rotate=45] at (xentp14) (plus) {{\footnotesize$+$}};
	\node [sum, below=1.0cm of entp13, draw] (xentp13) {};	
	\node [rotate=45] at (xentp13) (plus) {{\footnotesize$+$}};
	\node [sum, below=1.0cm of entp12, draw] (xentp12) {};	
	\node [rotate=45] at (xentp12) (plus) {{\footnotesize$+$}};		
	\node [sum, below=1.0cm of entp11, draw] (xentp11) {};	
	\node [rotate=45] at (xentp11) (plus) {{\footnotesize$+$}};
	\node [sum, below=1.0cm of entp10, draw] (xentp10) {};	
	\node [rotate=45] at (xentp10) (plus) {{\footnotesize$+$}};
	\node [sum, below=1.0cm of entp9, draw] (xentp9) {};	
	\node [rotate=45] at (xentp9) (plus) {{\footnotesize$+$}};
	\node [sum, below=1.0cm of entp8, draw] (xentp8) {};	
	\node [rotate=45] at (xentp8) (plus) {{\footnotesize$+$}};	

	\node [shape=dff,fill=cip, below=1cm of xentp15, draw] (ip7) {XX};	
	\node [shape=dff,fill=cip, below=1cm of xentp14, draw] (ip6) {XX};
	\node [shape=dff,fill=cip, below=1cm of xentp13, draw] (ip5) {XX};
	\node [shape=dff,fill=cip, below=1cm of xentp12, draw] (ip4) {XX};
	\node [shape=dff,fill=cip, below=1cm of xentp11, draw] (ip3) {XX};
	\node [shape=dff,fill=cip, below=1cm of xentp10, draw] (ip2) {XX};
	\node [shape=dff,fill=cip, below=1cm of xentp9, draw] (ip1) {XX};
	\node [shape=dff,fill=cip, below=1cm of xentp8, draw] (ip0) {XX};
	\node[below=1mm of ip4] (lip) {\large $in:$ip (8 byte)};	

	%%%%%% LINES >>
	
	% jiffies  -> c_low - lines
	\draw [->] (jiff3.south) -- (xjlch3.north);
	\draw [->] (jiff2.south) -- (xjlch2.north);	
	\draw [->] (jiff1.south) -- (xjlch1.north);		
	\draw [->] (jiff0.south) -- (xjlch0.north);		

	% c_high -> xjlch
	\draw [->] (chigh3.south)|- (xjlch3.west);
	\draw [->] (chigh2.south)|- (xjlch2.west);
	\draw [->] (chigh1.south)|- (xjlch1.west);
	\draw [->] (chigh0.south)|- (xjlch0.west);			

	% jiffies high -> j_high
	\draw [->] (jiff7.south) -- (jiffh3.north);
	\draw [->] (jiff6.south) -- (jiffh2.north);	
	\draw [->] (jiff5.south) -- (jiffh1.north);		
	\draw [->] (jiff4.south) -- (jiffh0.north);		

	% cycles_high -> c_high
	\draw [->] (cycl7.south) -- (chigh3.north);
	\draw [->] (cycl6.south) -- (chigh2.north);	
	\draw [->] (cycl5.south) -- (chigh1.north);		
	\draw [->] (cycl4.south) -- (chigh0.north);		
	
	% cycles_low -> c_high
	\draw [->] (cycl3.south) -- (xcli3.north);
	\draw [->] (cycl2.south) -- (xcli2.north);	
	\draw [->] (cycl1.south) -- (xcli1.north);		
	\draw [->] (cycl0.south) -- (xcli0.north);	

	% jiffies high -> j_high
	\draw [->] (jiffh3.south) -- (xjhi3.north);
	\draw [->] (jiffh2.south) -- (xjhi2.north);	
	\draw [->] (jiffh1.south) -- (xjhi1.north);		
	\draw [->] (jiffh0.south) -- (xjhi0.north);
		
	% xjhi -> irq
	\draw [<->] (xjhi3.south) -- (irq3.north);
	\draw [<->] (xjhi2.south) -- (irq2.north);	
	\draw [<->] (xjhi1.south) -- (irq1.north);		
	\draw [<->] (xjhi0.south) -- (irq0.north);
	
	% xjlch -> xentp
	\draw [->] (xjlch3.south) -- (xentp7.north);
	\draw [->] (xjlch2.south) -- (xentp6.north);	
	\draw [->] (xjlch1.south) -- (xentp5.north);		
	\draw [->] (xjlch0.south) -- (xentp4.north);

	% irq -> xcli
	\draw [->] (irq3.south) |- (xcli3.west);
	\draw [->] (irq2.south) |- (xcli2.west);	
	\draw [->] (irq1.south) |- (xcli1.west);		
	\draw [->] (irq0.south) |- (xcli0.west);

	% xentp -> entp
	\draw [<->] (xentp7.south) |- ($(entp7.north)!1/2!(entp7.north |- xentp7.south)$) coordinate (C1) -| (entp7.north);			
	\draw [<->] (xentp6.south) |- ($(entp6.north)!1/2!(entp6.north |- xentp6.south)$) coordinate (C2) -| (entp6.north);		
	\draw [<->] (xentp5.south) |- ($(entp5.north)!1/2!(entp5.north |- xentp5.south)$) coordinate (C3) -| (entp5.north);				
	\draw [<->] (xentp4.south) |- ($(entp4.north)!1/2!(entp4.north |- xentp4.south)$) coordinate (C4) -| (entp4.north);	
	
	% xcli -> xentp
	\draw [->] (xcli3.south) -- (xentp3.north);
	\draw [->] (xcli2.south) -- (xentp2.north);	
	\draw [->] (xcli1.south) -- (xentp1.north);		
	\draw [->] (xcli0.south) -- (xentp0.north);	

	% xcli -> xentp
	\draw [<->] (xentp3.south) -- (entp3.north);
	\draw [<->] (xentp2.south) -- (entp2.north);	
	\draw [<->] (xentp1.south) -- (entp1.north);		
	\draw [<->] (xentp0.south) -- (entp0.north);
	
	% xentp -> entp
	\draw [<->] (xentp15.north) -- (entp15.south);
	\draw [<->] (xentp14.north) -- (entp14.south);
	\draw [<->] (xentp13.north) -- (entp13.south);
	\draw [<->] (xentp12.north) -- (entp12.south);
	\draw [<->] (xentp11.north) -- (entp11.south);
	\draw [<->] (xentp10.north) -- (entp10.south);
	\draw [<->] (xentp9.north) -- (entp9.south);
	\draw [<->] (xentp8.north) -- (entp8.south);
	
	% ip -> xentp
	\draw [->] (ip7.north) -- (xentp15.south);
	\draw [->] (ip6.north) -- (xentp14.south);
	\draw [->] (ip5.north) -- (xentp13.south);
	\draw [->] (ip4.north) -- (xentp12.south);
	\draw [->] (ip3.north) -- (xentp11.south);
	\draw [->] (ip2.north) -- (xentp10.south);
	\draw [->] (ip1.north) -- (xentp9.south);
	\draw [->] (ip0.north) -- (xentp8.south);
	
	%%%%%% Legende 
\begin{scope}
	\node [shape=dff,fill=cip, below=10mm of entp6, draw] (f8) {};
	\node [sum, right=5mm of f8, draw] (xf8) {};	
	\node [rotate=45] at (xf8) (plus) {{\footnotesize$+$}};	
	\node [shape=dff,fill=cirq, right=7mm of xf8, draw] (c3) {};
	\draw [->] (f8.east) -- (xf8.west);
	\draw [<->] (xf8.east) -- (c3.west);
	\node[black, below=3mm of xf8, align=left] (lf81) {XOR-Operation applied on};	
	\node [shape=dff,fill=cip, right=0mm of lf81, draw] (rf8) {};
	\node[black, right=0mm of rf8, align=left] (lf82) {and};	
	
	\node [shape=dff,fill=cirq, below=22mm of entp6, draw] (yf8) {};
	\node[black, right=0mm of yf8, align=left] (lf83) {, result stored in};
	\node [shape=dff,fill=cirq, right=0mm of lf83, draw] (zf8) {};	
	\draw[black] ([xshift=-5mm, yshift=3mm ]f8.north west) rectangle ([xshift=23mm, yshift=-3mm]zf8.south east);	
\end{scope}	

\begin{scope}
\large
%\node [draw, align=center] {Text\\und Text};
\node[black, below=10mm of ip7, align=left] (lghdlbl) {\textbf{Label}};	
\node[black, right=28mm of lghdlbl.west, anchor=west, align=left] (lghddirr) {\textbf{Direction}};
\node[black, right=18mm of  lghddirr.west, anchor=west, align=left] (lghddesc) {\textbf{Description}};

%\node[black, below=10mm of ip4, align=left] (lgjif) {\textbf{cycles counter}};	
\node[black, below=4mm of lghdlbl.west, anchor=west, align=left] (lgcc) {cycles counter};	
\node[black, below=4mm of lgcc.west, anchor=west] (lgjif) {jiffies};	
\node[black, below=4mm of lgjif.west, anchor=west] (lgirq) {irq};	
\node[black, below=4mm of lgirq.west, anchor=west] (lgip) {ip};
\node[black, below=4mm of lgip.west, anchor=west] (lgentp) {fast pool};

\node[black, right=33mm of lgcc.west, align=left] (lgdircc) {in};	
\node[black, right=33mm of lgjif.west, align=left] (lgdirjif) {in};	
\node[black, right=33mm of lgirq.west, align=left] (lgdirirq) {in};
\node[black, right=33mm of lgip.west, align=left] (lgdirip) {in};
\node[black, right=30mm of lgentp.west, align=center] (lgdientp) {in/out};


\node[black, right=24mm of lgcc, align=left] (lgcctxt) {Number of CPU-cycles since system startup.};	
\node[black, below=4mm of lgcctxt.west, anchor=west] (lgjiftxt) {Number of ticks occured since system startup.};	
\node[black, below=4mm of lgjiftxt.west, anchor=west] (lgirqtxt) {Interrupt Request Code};
\node[black, below=4mm of lgirqtxt.west, anchor=west] (lgiptxt) {Instruction Pointer};
\node[black, below=4mm of lgiptxt.west, anchor=west] (lgentptxt) {Pseudo entropy pool, initializing input- \& nonblocking Pool};

\draw[black] ([xshift=-5mm, yshift=3mm ]lghdlbl.north west) rectangle ([xshift=5mm, yshift=-3mm]lgentptxt.south east);
	
\end{scope}
\end{tikzpicture}
%\caption{Pre-processing of input parameters by function 'add\_interrupt\_randomness' before applying further operations via fast\_mix / mix\_pool\_bytes (valid for x64 / 64-bit Kernel only)} \label{fig:add-int-rnd}
%\end{figure}

\begin{figure}[H]
	\centering
	\begin{tikzpicture}[scale=0.8, every node/.style={scale=0.8},font=\sffamily,>=triangle 45]
	\tikzstyle{sum} = [draw, shape=circle, node distance=1.5cm, line width=1pt, minimum width=1.25em]	
	%\Huge
	\def\N{7}  % Number of Flip-Flops minus one
	\def\BW{1.8} % Byte Width
	
	% colors
	\definecolor{cjh}{HTML}{CFD2D4}
	\definecolor{cjl}{HTML}{DD6262}
	\definecolor{cch}{HTML}{F7A100}
	\definecolor{ccl}{HTML}{0288CF}
	\definecolor{cep32}{HTML}{64FE2E}
	\definecolor{cep1}{HTML}{1893D4}
	\definecolor{cep0}{HTML}{A9FD2A}
	\definecolor{cirq}{HTML}{DB15E5}
	\definecolor{cip}{HTML}{FFF200}
	
	% rect rc-inp-pid
	\node (rc-inp-pid) [rectangle, fill=cjh, draw, minimum width=35mm, minimum height=10mm, anchor= south west] at (0,0) {\shortstack{$in:$pid\\(4 byte))}};
	% rect rc-inp-jiffies
	\node (rc-inp-jiffies) [rectangle, right=10mm of rc-inp-pid, fill=cjh, draw, minimum width=35mm, minimum height=10mm, anchor= west] {\shortstack{$in:$jiffies\\(4/8 byte)}};	
	% rect rc-inp-cycles	
	\node (rc-inp-cycles) [rectangle, right=10mm of rc-inp-jiffies, fill=cjh, draw, minimum width=35mm, minimum height=10mm, anchor= west] {\shortstack{$in:$cycles\\(4/8 byte)}};	
	% oa-pjc - add pid/jiffies/cycles
	\node [sum, below=10mm of rc-inp-jiffies, draw] (oa-pjc) {};	
	\node at (oa-pjc) (plus) {{\footnotesize$+$}};	
	% rc-inp-pid -> oa-pjc
	\draw [->] (rc-inp-pid) |- (oa-pjc);	
	% rc-inp-jiffies -> oa-pjc
	\draw [->] (rc-inp-jiffies) -- (oa-pjc);
	% rc-inp-cycles -> oa-pjc
	\draw [->] (rc-inp-cycles) |- (oa-pjc);
	% ox-oa-pjc-hash 
	\node [sum, below=10mm of oa-pjc, draw] (ox-oa-pjc-hash) {};	
	\node [rotate=45] at (ox-oa-pjc-hash) (plus) {{\footnotesize$+$}};	
	% oa-pjc -> ox-oa-pjc-hash
	\draw [->] (oa-pjc) -- (ox-oa-pjc-hash);	
	% rc-hash	
	\node (rc-hash) [rectangle, below=10mm of ox-oa-pjc-hash, fill=cjh, draw, minimum width=35mm, minimum height=10mm] {\shortstack{$in/out:$hash}};		
	% ox-oa-pjc-hash <-> md5-transf
	\draw [<->] (ox-oa-pjc-hash) -- (rc-hash);	
	% md5-transf
	\node (md5-transf) [rectangle, below=10mm of rc-hash, fill=cjh, draw, minimum width=35mm, minimum height=10mm, rounded corners=0.5cm] {\shortstack{MD5 Transform}};	
	% rc-hash <-> md5-transf
	\draw [<->] (rc-hash) -- (md5-transf);
	% rect rc-rnd-int-sec
	\node (rc-rnd-int-sec) [rectangle, right=19mm of ox-oa-pjc-hash, fill=cjh, draw, minimum width=35mm, minimum height=10mm, anchor= west] {\shortstack{$in:$random int secret\\(16 byte)}};	
	% rc-rnd-int-sec <-> md5-transf
	\draw [->] (rc-rnd-int-sec) |- (md5-transf);
	
	%%%%%%%%% Legende
	\begin{scope}
	
	% ox
	\node [sum,  below right=25mm and 18mm of rc-inp-pid.west, draw] (loa) {};	
	\node at (loa) (plus) {{\footnotesize$+$}};	
	\node[black, right=1mm of loa.east, anchor=west, align=left] (lloa) {ADD Operation};		
	% oa
	\node [sum, below=2mm of loa, draw] (lox) {};	
	\node [rotate=45] at (lox) (plus) {{\footnotesize$+$}};	
	\node[black, right=1mm of lox.east, anchor=west, align=left] (llox) {XOR Operation};		

	\draw[black] ([xshift=-5mm, yshift=3mm ]loa.north west) rectangle ([xshift=5mm, yshift=-3mm]llox.south east);	
	
	\end{scope}
		
	\begin{scope}	
	\large
	%\node [draw, align=center] {Text\\und Text};
	\node[black, below=76mm of rc-inp-pid.west, align=left] (lghdlbl) {\textbf{Label}};	
	\node[black, right=28mm of lghdlbl.west, anchor=west, align=left] (lghddirr) {\textbf{Direction}};
	\node[black, right=18mm of  lghddirr.west, anchor=west, align=left] (lghddesc) {\textbf{Description}};
	
	%\node[black, below=10mm of ip4, align=left] (lgjif) {\textbf{cycles counter}};	
	\node[black, below=4mm of lghdlbl.west, anchor=west, align=left] (lgcc) {cycles};	
	\node[black, below=4mm of lgcc.west, anchor=west] (lgjif) {jiffies};	
	\node[black, below=4mm of lgjif.west, anchor=west] (lgirq) {pid};	
	\node[black, below=4mm of lgirq.west, anchor=west] (lgip) {random int secret};
	\node[black, below=4mm of lgip.west, anchor=west] (lgentp) {hash};
	
	\node[black, right=33mm of lgcc.west, align=left] (lgdircc) {in};	
	\node[black, right=33mm of lgjif.west, align=left] (lgdirjif) {in};	
	\node[black, right=33mm of lgirq.west, align=left] (lgdirirq) {in};
	\node[black, right=33mm of lgip.west, align=left] (lgdirip) {in};
	\node[black, right=30mm of lgentp.west, align=center] (lgdientp) {in/out};
	
	
	\node[black, right=46mm of lgcc.west, align=left] (lgcctxt) {Number of CPU-cycles since system startup.};	
	\node[black, below=4mm of lgcctxt.west, anchor=west] (lgjiftxt) {Number of ticks occured since system startup.};	
	\node[black, below=4mm of lgjiftxt.west, anchor=west] (lgirqtxt) {Process Id};
	\node[black, below=4mm of lgirqtxt.west, anchor=west] (lgiptxt) {Hidden entropy buffer, once initialized at startup};
	\node[black, below=4mm of lgiptxt.west, anchor=west] (lgentptxt) {Kernel entropy buffer, transformed with each call};
	
	\draw[black] ([xshift=-5mm, yshift=3mm ]lghdlbl.north west) rectangle ([xshift=5mm, yshift=-3mm]lgentptxt.south east);
	
	\end{scope}	

\end{tikzpicture}	
	\caption{Processing of input parameters by function 'get\_random\_int/long' (valid for x64 / 64-bit Kernel only)} \label{fig:get-rnd-int-long}
\end{figure}



%\begin{mdframed}
%\begin{tabularx}{\columnwidth}{XXl}
%\begin{tabularx}{\textwidth}{ll}
%%	\caption{Description of input parameters proccessed by func. 'add\_interrupt\_randomness'}
%%	\label{tab:add-int-rnd-desc}\\
%	\textbf{jiffies}&Igel\\
%	\textbf{cycles counter}&Dienstag\\
%	\textbf{irq}&\\
%	\textbf{ip}&\textit{Instruction Pointer}
%	\caption{Description of input parameters proccessed by func. 'add\_interrupt\_randomness'}
%\end{tabularx}
%\centering
%\begin{table}[H]
%	\begin{tabular}{ll}
%	\textbf{Jiffies}&Nr. of ticks occured since system startup.\\
%	\textbf{Cycles Counter}&Nr. of CPU-cycles since system startup.\\
%	\textbf{irq}&Interrupt Request \\
%	\textbf{ip}&\textit{Instruction Pointer}
%	\end{tabular}
%	\caption{Description of input parameters proccessed by func. 'add\_interrupt\_randomness'}	
%\end{table}

% jiffies . Incremented for each timer interrupt.
% . also known as \textit{Time Stamp Counter}.

%	\cite{kernlrandmc}
		
%\end{mdframed}


%\begin{tabularx}{\columnwidth}{XXl}
%	jiffies&Schnecke&Igel\\
%	cycles counter&Hier ist ein langes Wort Hier ist ein langes Wort Hier ist ein langes Wort Hier ist ein langes Wort &Dienstag\\
%	irq&&\\
%	ip&&
%\end{tabularx}
%
%
%
%\begin{mdframed}
%\begin{description}
%	\item [jiffies] sadfasdfsadf
%	\item [cycles counter] asdfasdfasdf
%	\item [irq] asdfasdfasdf	
%	\item [ip] 
%\end{description}
%\cite{kernlrandmc}	
%\end{mdframed}




\section{Conclusion}
The final section of the chapter gives an overview of the important results
of this chapter. This implies that the introductory chapter and the
concluding chapter don't need a conclusion.

%\lipsum[66]

%%% Local Variables: 
%%% mode: latex
%%% TeX-master: "thesis"
%%% End: 
