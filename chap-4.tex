\chapter{NIST Special Publication 800-90B}
\label{chap:NIST800-90B}
\section{Generall}

Topic of this thesis is an analysis of a Linux Random Number Generator's outcome delivering values to kernel consumers introduced in \ref{sub:get-rnd-int-long}. Verification and estimation of entropy is 
a very difficult task, since there is no explicit definition for randomness. Hence, for a series of values generated by an RNG, randomness i.e. a sufficient degree of entropy is admitted if the absence of any pattern within records can be proven. 



\cite{turan2015random}

Estimating entropy is a difficult (if not impossible) problem, and we've been working to create
usable guidance that will give conservative estimates on the amount of entropy in an entropy
source.
\cite{turan2015random}

%-----------------------------------------------
%NIST Special Publication (SP) 800-90 (a series consisting of three documents) is all about
%generating random numbers for cryptography. In SP 800-90, this is a two-stage process: first, an
%entropy source provides an impossible-to-guess seed. Then, a deterministic cryptographic
%algorithm (called a DRBG--deterministic random bit generator--in SP 800-90) expands the seed
%into a long sequence of values that may be safely used for keys, IVs, nonces, etc.
%\cite{turan2015random}

\section{Terminology}
In Information theory, for a set of information units, entropy in generall describes the level of disorder i.e. the unpredicatbility of each single unit's state at a given alphabet. Various formal defintions of entropy exist \cite{hagerty2012entropy}, while the definition of Shannon (see \ref{fig:form-entropy-shan}) is quite common for a binary alphabet. 

\begin{figure}[H]
	\begin{align*}
	\displaystyle H(X) := \sum_{s \in {0,1}} I(p_s) p_s = \sum_{s \in {0,1}} -\lg_2 \frac{1}{p_s} p_s && \text{$p_s$ = P(X=s)}
	\end{align*}
	\caption{Formal definiton of entropy by Shannon for binary alphabet}
	\label{fig:form-entropy-shan}
\end{figure}
According to Shannon, entropy is defined as the product of self-information \textit{I} multiplied by the expected value. For a binary alphabet, it can be measured in unit bits. 
