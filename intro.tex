\chapter{Introduction}
\label{cha:intro}
\section{Motivation}
Operating systems implement a variety of security components which depend significantly on the entropy of random numbers. In general, those random numbers can be provided by operating system internal or external providers, so called Random Number Generators (RNG). Typical external providers are hardware devices, able to deviate random numbers from radioactive decay, atmospheric or thermal noise, photoelectric effects etc. Since those numbers are assumed to provide a high level of entropy i.e. random numbers of high quality, they are also referred as True Random Number Generators (TRNG). Contrary, RNGs implemented as part of an operating system, the lack of comparable stochastic processes and hence are labeled as Pseudo Random Number Generator (PRNG). To compensate this lack, PRGN implementations need to fall back on alternative sources, able to provide information with a certain degree of entropy. Common approaches of operating systems usually adopt information from external sources, f.e. user input like keyboard events or mouse coordinates. Beside further external input like network traffic, system internal individual information like serial numbers of hardware devices, MAC addresses or the occurrence of hardware interrupts is used to gain an entropy pool of pseudo random numbers. Consequently, the process of generating random numbers by an operating system PRNG is influenced by the way the system interacts with its environment i.e. how it is set up. This aspect should be considered when applying server operating systems as virtual machines i.e. in virtual environments. \\~\\
Virtualization of operating systems has increased continuously over the recent years and layed the foundations for new resource management concepts and business models like f.e. cloud hosting. Regarding server virtualization, many organizations exceed rates of 75\% \cite{gartnervmmarket}. By executing an operating system as a virtual machine, improvements in several areas, like scalability and more efficient utilization of hardware resources, backup \& disaster recovery strategy, accelerated provision of new system instances etc. can be achieved. Depending on the applied technology, such virtual machines can be setup up on basis of operating system setup images which are targeted for installation directly on hardware, without any further modification required. In other words, the virtual machine monitor or hypervisor enables the execution of regular commercial or open source operating systems in a virtual environment. However, while there are many reasons to utilize virtual machines, most of the standard security practices of operating systems are based on assumptions that hold true for physical machines, but don't translate immediately into the domain of virtualized machines \cite{kerrigan2012study}. A virtual hosted Linux server f.e. will never receive a user triggered event during the boot process, which is a critical period within the initialization of the internal Linux PRNG. Further, new virtual machine instances are usually provided by cloning a master image, instead of running an installation setup, leading to a higher degree of homogeneity among virtual machine instances. While provisioning time and maintenance benefit from this practice, it may be regarded critical. If this approach f.e. is applied as the standard process of a cloud hosting provider, it`s customers may be delivered a vulnerable system from the outset, since one customer might gain insights regarding another customers system, just by analyzing his own. These insights can be used to exploit VM reset vulnerabilities which take advantage of the reuse of  operating system snapshots, so called snapshot replay. Thomas Ristenpart et. al. describe this type of attack and apply it on TSL implementations with disastrous results. Relevant reasons they identified are the exposure of randomness, as well as the inability to find sufficient entropy in virtual systems environments \cite{ristenpart2010good, ristenpart2009hey}. \\~\\
According to a study of Gartner from 2016, the market for x86 server virtualization infrastructure software is partitioned among few competitors, led by VMWare, Microsoft, Oracle and Red Hat. While VMWare and Microsoft offer proprietary solutions, Oracle and Red Hat adopted the open-source Xen-hypervisor technology. For cloud infrastructures, the Xen-hypervisor remains the most widely used architecture for public infrastructure as a service (IaaS) cloud provider. This fact is predominantly attributable to Amazon's utilization of the Xen-hypervisor for its cloud solutions "Amazon Web Services" \cite{bittman2016magic}. Similarly, the market of common server operating systems can be divided into proprietary and open-source solutions. While Microsoft dominates the fraction of proprietary systems with its Windows Server series, Linux based systems have a total share of \~37\%. Within this share, \~39\% of servers dedicated to host websites etc. are running an Ubuntu Distribution \cite{statsharelinux} . \\~\\
Summarized, recent studies have revealed that virtualization of Linux based servers is widespread, while the execution of those systems as virtual machine may bring up critical vulnerabilities. Based on these considerations the objective of this thesis is an analysis of the generation of random numbers by a Linux 4.X.X.X kernel's internal PRNG in a Xen-hypervisor environment. To this date, several research contributions have been published in this field. Those may be divided by their scope regarding the Linux PRNG responsibilities, attack vectors and, if analyzing the entropy pools state, the approach of assessing randomness. While some papers address the PRNG output for the entire time of a system's uptime, the focus of this research will be a system's startup i.e. boot phase. This period is meant to be of certain interest for several reasons: The initialization of the Linux PRMG is beginning as one of the first activities conducted when a (virtual) machine starts to bring up the systems kernel, as it is eager to use as many noise sources as possible to build up various entropy pools. Further, during this initialization process, some buffers are assigned with values which are valid and reused over the entire uptime of the system. Poorly developed values for these buffers are assumed to weaken a number of security components and hence are of particular interest. Finally, a part of the boot process is the startup of various daemon processes, which usually remain in execution over the entire system uptime. Beside daemons required by the system, this may also be applications like web-, ftp-, ssh-, proxy-servers which may be reachable via the internet and thus potentially vulnerabilities should be classified as critical.
To obtain a comprehensive data record for evaluation, a Linux 4.X.X.X kernel has been enhanced by a tracking feature, able to record information at points of interest, without producing any noise itself. The feature is not just able to track the values returned to applications requesting random values, but also to breakdown how those have been generated over several stages until their origin in terms of noise. This kernel is brought up within an Ubuntu Server 16.04 system, hosted as a guest on Xen hypervisor 4.XX. The Linux PRNG has an internal rating system describing an entropy pools quality in "bits of entropy". Some research contributions (f.e. \cite{lacharme2012linux})refer to this system when assessing random number quality. To conduct an independent and fine grained analysis, for this thesis a Statistical Test Suite, published by the NIST has been applied \cite{paul2016nist}. Various operating systems security concepts rely on solid random numbers. Hence, besides the processing of noise to random numbers, also the consequences of weak random numbers have been analyzed on the effectiveness of two fundamental operating system security technologies: Stack Guards/Stack Canaries and Address Space Randomization Layout (ASLR).
The following sections will deliver the required background knowledge regarding virtualization via the Xen-hyperisor, the architecture and generation of the Linux Pseudo Random Number Generator, the concepts of stack canaries and ASLR and Statistical Random Number Validation. 



\section{Related Reserach}









% Virtualization in these terms means, that an operating system is not directly installed on a hardware host. Instead on the host machine an application is installed beeing able administer systems resources and assign those to several operating systems which may be run on host simultaneously.  

   





%%% Local Variables: 
%%% mode: latex
%%% TeX-master: "thesis"
%%% End: 
